%midterm-report.tex - Report for CS 425
\documentclass[twoside]{report}
\usepackage{listings}
\lstset{frame=tb,language=C++,numbers=none}
\begin{document}
  \begin{titlepage}
    \title{Midsemester Report on Aruco Markers using the Rasperry Pi 3 and Opencv 3.2}
    \date{March 19, 2017}
    \author{Ryan Owens}
    \maketitle
  \end{titlepage}
  \section{Camera Callibration and Marker Detection}
    Thus far I have spent most of my time working with camera callibration. My goal
    for camera callibration is to provide an easy-to-use class that will grab
    frames from a camera and calculate the distortions.  The current version of
    my camera callibration class requires 15 frames to be captured using a usb
    camera. During the callibration, my class will ask the user for the 15 frames.
    Example code for calling the calibration is as follows:
    \begin{lstlisting}
#include "CameraCalibrator.h"

int main(int argc, char const *argv[]) {
  // Instantiate a new callibrator providing a camera id of 1
  // and setting vertical flip of captured images to true
  CameraCalibrator *calibrator = new CameraCalibrator(1, true);

  // Run the callibration
  calibrator->calibrate();
  return 0;
}
    \end{lstlisting}
    The calibrate function runs through the entire camera callibration and saves
    both translation and rotation distortions into a file.

      With camera distortion values obtained, aruco marker detection and pose
    estimation can be performed. An ArucoDetector class has been implemented which
    provides an easy means for finding markers.  Initial goals of the class are
    to allows developers the ability to check if a particular marker is in view,
    to get an array of all markers currently in view and to get the estimated
    pose of a marker relative to the camera's position. As of now the ArucoDetector
    class can perform all three functions.  An example of how to use the class is
    as follows:
    \begin{lstlisting}
#include <iostream>
#include "ArucoDetector.h"

using namespace std;

int main(int argc, char** argv)
{
// Instantiate a new ArucoDetector providing a camera
// id of 1 and turning Vertical Image flip to false
ArucoDetector* detector = new ArucoDetector(1, false);
// opening the specified camera
detector->openCamera();
cout << "Running through marker in view" << endl;
for(int i = 0; i < 10; i++)
{
  // Checking if the marker with id i is in view
	if(detector->arucoMarkerInView(i))
	cout << "Marker " << i << " is in view" << endl;
	else
	cout << "Marker " << i << " is not in view" << endl;
}
cout << endl << "Running through markers in view" << endl;
// Getting an array of markers currently in view
vector<int> markers = detector->arucoMarkersInView();
for(auto itr = markers.begin(); itr != markers.end(); ++itr)
{
	cout << "Marker " << *itr <<  " is in view." << endl;
  // Getting the pose of a marker by its id
	vector<double> rottransvec = detector->getPose(*itr);
	cout << endl << "Running get pose for id " << *itr << endl;
	for(auto itr = rottransvec.begin(); itr != rottransvec.end(); ++itr)
	{
		cout << *itr << endl;
	}
}
// Getting the pose of a marker by its id
// during testing, this marker was not in view
// and the expected pose was x=0, y=0, z=0,
// rotx=0, roty=0, rotz=0
vector<double> rottransvec = detector->getPose(5);
cout << endl << "Running get pose for id " << 5 << endl;
for(auto itr = rottransvec.begin(); itr != rottransvec.end(); ++itr)
{
	cout << *itr << endl;
}
return 0;
}
  \end{lstlisting}
  Currently, arrays are substituted with c++ standard vectors. I have been
  considering an interface for the pose estimation. This interface would be
  its own class that provides the relevent data as needed. However, without
  knowing exactly how KIPR plans to use the pose data it has been difficult
  to plan a proper interface class. I'm also planning to extend the ArucoDetector
  class with support for Aruco Diamond markers.
  \section{Marker Dictionary Generation}
  Dictionary generation is very simple using the aruco dictionary generator function.
  The biggest obstacle with the dictionary generation is saving the dictionary for
  later use. OpenCV's FileStorage class makes saving and retrieving the dictionary
  easy.  Example code for generating a custom dictionary is as follows:
  \begin{lstlisting}
#include "DictionaryGenerator.h"
#include <iostream>

using namespace std;

bool compareDictionaries(aruco::Dictionary *dict1, aruco::Dictionary *dict2) {
  return (dict1->markerSize == dict2->markerSize) &&
         (dict1->maxCorrectionBits == dict2->maxCorrectionBits);
}

int main(int argc, char const *argv[]) {
  // Instantiate a Dictionary Generator with
  // The number of markers in the dictionary is 16
  // The dimensions of the markers is 20
  DictionaryGenerator *dictionaryGenerator = new DictionaryGenerator(16, 20);
  // Generate the new Dictionary
  auto dictionary = dictionaryGenerator->generateNewDictionary();
  // Write the Dictionary to file
  dictionaryGenerator->saveDictionary(&dictionary);
  // Read the Dictionary from file
  auto retrievedDictionary =
      dictionaryGenerator->getDictionaryFromFile("kipr_dictionary.yml");
  if (compareDictionaries(&dictionary, retrievedDictionary))
    cout << "Dictionaries are the same" << endl;
  else
    cout << "Dictionaries are not the same" << endl;
  return 0;
  }
  \end{lstlisting}
  \section{GPU Acceleration Possibilities}
  At the beginning of the semester I beleived that GPU acceleration libraries
  were available for the Raspberry Pi 3.  As it turns out GPU acceleration on the
  Broadcom VideoCore IV, the GPU in the Pi 3, has not fully been implemented as
  documentation for the GPU was released in 2014. Since the release it seems that
  GPU acceleration has not been attempted other than the usual DOOM port.

  Without some form of GPU acceleration library, any GPU work on the Raspberry Pi
  3 will not be feasible. I have my doubts as to whether or not GPU acceleration
  on the Pi 3 is worth putting any time or effort into.  The VideoCore IV has only
  12 QPU cores and an L2 cache of 128Kb shared accros 3 QPU slices. My biggest
  concern with GPU acceleration on the Pi 3 is sharing memory between the GPU and
  CPU since the GPU does not have dedicated memory. In order to perform any worth
  while calculations on the GPU it should be given at least 256Mb of memory.  This
  reduces the memory available to the CPU by a fourth.

  Shared memory between the CPU and GPU does, however, reduce the latency between
  copying data to and from the GPU as it does not have to be obtained from GPU
  memory. Considerations must also be made for the design of the GPU as a low
  power SIMD based processor; it is just not that powerful.
  \section{Plan for the rest of the semester}
  I am hoping to spend the rest of the semester working on libwallaby integration
  as well as further research on possible GPU acceleration for the Raspberry Pi 3.
  In the next few weeks I will begin benchmarking the Pi 3 GPU and compare different
  memory configurations.  Then, using the benchmark results, I will document my
  theory on GPU acceleration on the Pi 3 platform.  Meanwhile I am also planning
  to continue my support of Aruco libwallaby integration. 
\end{document}
